\documentclass[11pt, a4paper]{article}

\usepackage{a4wide}
\usepackage{xcolor}
\usepackage{graphicx}
\usepackage[T1]{fontenc}
\usepackage[utf8]{inputenc}
%\usepackage[english]{babel}
\usepackage{csquotes}
\usepackage{mathptmx}
%\usepackage{MnSymbol}
%\usepackage{pifont}
\usepackage{amsmath,amsthm,amssymb}
\usepackage{soul}
\usepackage{wrapfig}
\usepackage{mathtools}
\usepackage{hyperref}


%\usepackage[style=numeric-comp, sorting=none]{biblatex}
%\addbibresource[glob]{~/Repositories/papers/short-papers/bib/bibsysbiol.bib}
%\addbibresource{bibsysbiol.bib}

\setlength{\parindent}{0pt}
\setlength{\parskip}{1ex plus 0.5ex minus 0.2ex}

\title{BINF200 H2025 learning goals}

\author{Tom Michoel}

\date{\today}

\begin{document}

\maketitle

\section{Molecular biology}

Know essential concepts of molecular biology necessary for understanding biological sequences and structures.

\begin{itemize}
    \item Be able to explain the central dogma
    \item Be able to exaplain what base pairing, genes, introns, exons, codons, etc. are.
    \item Be able to explain what the genetic code is.
    \item Be able to construct the complimentary sequence of a given sequence.
    \item Be able to to translate an RNA sequence into a protein sequence using a genetic code table.
\end{itemize}

\section{Global pairwise sequence alignment}

Understand the Needleman-Wunsch algorithm.

\begin{itemize}
    \item Be able to explain the principle of dynamic programming and why it can be used to maximize a pairwise alignment score.
    \item Be able to fill a dynamic programming table given a scoring scheme.
    \item Be able to backtrack in a dynamic programming table to find one or more optimal global alignments.
\end{itemize}

\section{BLAST}

Understand the basic steps of the BLAST algorithm.

\begin{itemize}
    \item Be able to compile a list of n-grams for a given input sequence.
    \item Be able to create a look-up table of n-grams for a given input sequence.
    \item Be able to create a score table for matching n-grams for a given score matrix.
    \item Be able to create a look-up table of matching n-grams.
    \item Be able to search a database for sequences with matching n-grams using a score threshold.
    \item Be able to explain the concept of statistical significance and the E-value.
\end{itemize}

\section{Multiple sequence alignment}

Understand the principle and main approach of multiple sequence alignment (MSA).

\begin{itemize}
    \item Be able to explain different MSA scoring methods (sum of pairs, entropy-based).
    \item Be able to explain why a dynamic programming solution exists in theory but is not practical.
    \item Be able to explain and apply the progressive multiple sequence alignment algorithm.
\end{itemize}

\section{Phylogenetics}

Understand what is phylogenetics and how phylogenetic trees are constructed.

\begin{itemize}
    \item Be able to explain what evolutionary relationships are.
    \item Be able to explain the difference between physiological trait-based and molecular sequence-based phylogenetics.
    \item Be able to explain the major assumptions of molecular phylogenetics.
    \item Be able to define the root of a tree using an outgroup or midpoint rooting.
    \item Be able to explain the different types of trees.
    \item Be able to explain the different steps in tree construction.
    \item Be able to explain and apply the different steps of the UPGMA algorithm for constructing a tree from a matrix of pairwise distances.
    \item Be able to explain and apply the principle of maximum parsimony.
\end{itemize}

\section{Sequence motifs}

Understand what sequence motifs are, how known motifs can be detected in a sequence, and how new motifs can be discovered.

\begin{itemize}
    \item Be able to give examples of types of regulatory sites in a genome and their biological function.
    \item Be able to explain what position-specific count, probability, and log-odds score matrices are.
    \item Be able to detect known motifs in a sequence by scoring all sequence positions using a position-specific probability or log-odds score matrix.
    \item Be able to explain the concept of statistical significance for deciding a score cutoff.
    \item Be able to explain the motif discovery problem and why it is hard.
    \item Be able to explain the Expectation-Maximization algorithm, incl.\ the general idea of the algorithm, the main mathematical equations or principles defining the steps of the algorithm, and high-level pseudo-code.
    \item Be able to explain the Gibbs sampler algorithm, incl.\ the general idea of the algorithm, the main mathematical equations or principles defining the steps of the algorithm, and high-level pseudo-code.
\end{itemize}

\section{Hidden Markov models}

Understand what hidden Markov models (HMMs) are, how the hidden state path can be inferred from a sequence of observations, and what HMMs are used for in biological sequence analysis.

\begin{itemize}
    \item Be able to explain what a Markov chain is, what a HMM is, and what the difference between them is.
    \item Be able to give a formal definition of a HMM.
    \item Be able to compute the joint probability of observing a sequence of states and emitted symbols given the state transition and emission probabilities of a HMM.
    \item Be able to explain the Viterbi algorithm for finding the most probable hidden state path from a sequence of observations, incl.\ the general idea of the algorithm, the main mathematical equations or principles defining the steps of the algorithm, and high-level pseudo-code.
    \item Be able to explain the general idea of posterior decoding and how it differs from finding the most probably state path.
    \item Be able to give examples of biological sequence analysis tasks that can be solved using HMMs.
\end{itemize}

\section{RNA secondary structure}

Understand the physical origin of RNA folding, how RNA secondary structure can be represented, and how folded structures can be predicted from an RNA sequences.

\begin{itemize}
    \item Be able to explain the physical origin of RNA folding.
    \item Be able to explain the assumption of no tertiary interactions and what it means for defining secondary structures.
    \item Be able to convert the secondary structure for a given RNA sequence into its dot-bracket notation, and vice versa.
    \item Be able to explain the Nussinov folding algorithm, incl.\ the general idea of the algorithm, the main mathematical equations or principles defining the steps of the algorithm, the main difference with standard dynamic programming algorithms for sequence alignment, and high-level pseudo-code.
    \item Be able to explain why maximizing the number of base pairs does not necessarily result in the thermodynamically most stable structure.
\end{itemize}

\section{Protein structure analysis}

Know fundamental structures of biological molecules, how structures are determined, and where structural information can be found.

\begin{itemize}
    \item Be able to explain the fundamental structure of DNA, RNA, and proteins.
    \item Be able to explain multiple mechanisms by which structures are tuned.
    \item Be able to list important knowledgebases to find information about genes, proteins, and protein structures and interactions.
    \item Be able to explain the basic building blocks of protein structures.
    \item Be able to list multiple fundamental experimental methods for protein structure determination.
    \item Be able to explain why we want and can predict protein tertiary structure from sequence.
    \item Be able to explain what is CASP.
    \item Be able to explain classical methods for protein tertiary structure prediction.
    \item Be able to explain what is alphafold and why it has revolutionized the field of protein tertiary structure prediction.
\end{itemize}

\end{document}