\documentclass[11pt, a4paper]{article}

\usepackage{a4wide}
\usepackage{xcolor}
\usepackage{graphicx}
\usepackage[T1]{fontenc}
\usepackage[utf8]{inputenc}
%\usepackage[english]{babel}
\usepackage{csquotes}
\usepackage{mathptmx}
%\usepackage{MnSymbol}
%\usepackage{pifont}
\usepackage{amsmath,amsthm,amssymb}
\usepackage{soul}
\usepackage{wrapfig}
\usepackage{mathtools}
\usepackage{hyperref}


%\usepackage[style=numeric-comp, sorting=none]{biblatex}
%\addbibresource[glob]{~/Repositories/papers/short-papers/bib/bibsysbiol.bib}
%\addbibresource{bibsysbiol.bib}

\setlength{\parindent}{0pt}
\setlength{\parskip}{1ex plus 0.5ex minus 0.2ex}

\title{BINF200 H2025 learning goals}

\author{Tom Michoel}

\date{\today}

\begin{document}

\maketitle

\section{Molecular biology}

Know essential concepts of molecular biology necessary for understanding biological sequences and structures.

\begin{itemize}
    \item Be able to explain the central dogma
    \item Be able to exaplain what base pairing, genes, introns, exons, codons, etc. are.
    \item Be able to explain the genetic code.
    \item Be able to construct the complimentary sequence of a given sequence.
    \item Be able to to translate a given RNA sequence into a protein sequence.
\end{itemize}

\section{Global pairwise sequence alignment}

Understand the Needleman-Wunsch algorithm.

\begin{itemize}
    \item Be able to explain the principle of dynamic programming and why it can be used to maximize a pairwise alignment score.
    \item Be able to fill a dynamic programming table given a scoring scheme.
    \item Be able to backtrack in a dynamic programming table to find one or more optimal global alignments.
\end{itemize}

\section{BLAST}

Understand the basic steps of the BLAST algorithm.

\begin{itemize}
    \item Be able to compile a list of n-grams for a given input sequence.
    \item Be able to create a look-up table of n-grams for a given input sequence.
    \item Be able to create a score table for matching n-grams for a given score matrix.
    \item Be able to create a look-up table of matching n-grams.
    \item Be able to searching a database for sequences with matching n-grams using a score threshold.
    \item Be able to explain the concept of statistical significance and the E-value.
\end{itemize}

\section{Multiple sequence alignment}

Understand the principle and main approach of multiple sequence alignment (MSA).

\begin{itemize}
    \item Be able to explain different MSA scoring methods (sum of pairs, entropy-based).
    \item Be able to explain why a dynamic programming solution exists in theory but is not practical.
    \item Be able to explain and apply the progressive multiple sequence alignment algorithm.
\end{itemize}

\section{Phylogenetics}

Understand what is phylogenetics and how phylogenetic trees are constructed.

\begin{itemize}
    \item Be able to explain what evolutionary relationships are.
    \item Be able to explain the difference between physiological trait-based and molecular sequence-based phylogenetics.
    \item Be able to explain the major assumptions of molecular phylogenetics.
    \item Be able to define the root of a tree using an outgroup or midpoint rooting.
    \item Be able to explain the different types of trees.
    \item Be able to explain the different steps in tree construction.
    \item Be able to explain and apply the different steps of the UPGMA algorithm for constructing a tree from a matrix of pairwise distances.
    \item Be able to explain and apply the principle of maximum parsimony.
\end{itemize}

\section{Sequence motifs}

Understand what sequence motifs are, how known motifs can be detected in a sequence, and how new motifs can be discovered.

\begin{itemize}
    \item Be able to give examples of types of regulatory sites in a genome and their biological function.
    \item Be able to explain what position-specific count, probability, and log-odds score matrices are.
    \item Be able to detect known motifs in a sequence by scoring position using a position-specific probability or log-odds score matrix.
    \item Be able to explain the concept of statistical significance for deciding a score cutoff.
    \item 
\end{itemize}

\end{document}