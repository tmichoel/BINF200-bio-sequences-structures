\documentclass[11pt, a4paper]{article}

\usepackage{a4wide}
\usepackage{xcolor}
\usepackage{graphicx}
\usepackage[T1]{fontenc}
\usepackage[utf8]{inputenc}
%\usepackage[english]{babel}
\usepackage{csquotes}
\usepackage{mathptmx}
%\usepackage{MnSymbol}
%\usepackage{pifont}
\usepackage{amsmath,amsthm,amssymb}
\usepackage{soul}
\usepackage{wrapfig}
\usepackage{mathtools}
\usepackage{hyperref}


%\usepackage[style=numeric-comp, sorting=none]{biblatex}
%\addbibresource[glob]{~/Repositories/papers/short-papers/bib/bibsysbiol.bib}
%\addbibresource{bibsysbiol.bib}

\setlength{\parindent}{0pt}
\setlength{\parskip}{1ex plus 0.5ex minus 0.2ex}

\title{BINF200 learning goals}

\author{Tom Michoel}

\date{\today}

\begin{document}

\maketitle

\section{Molecular biology}

Know essential concepts of molecular biology necessary for understanding biological sequences and structures.

\begin{itemize}
    \item The central dogma
    \item The genetic code
    \item Base pairing and complimentary sequences
\end{itemize}

\section{Global pairwise sequence alignment}

Understand and be able to apply the Needleman-Wunsch algorithm.

\begin{itemize}
    \item Understand the principle of dynamic programming and why it can be used to maximize a pairwise alignment score.
    \item Be able to fill a dynamic programming table given a scoring scheme.
    \item Be able to backtrack in a dynamic programming table to find one or more optimal global alignments.
\end{itemize}

\section{BLAST}

Understand the basic steps of the BLAST algorithm.

\begin{itemize}
    \item Compiling a list of n-grams for a given input sequence.
    \item Creating a look-up table of n-grams for a given input sequence.
    \item Creating a score table for matching n-grams for a given score matrix.
    \item Creating a look-up table of matching n-grams.
    \item Searching a database for sequences with matching n-grams using a score threshold.
    \item Understand the concept of statistical significance and the E-value.
\end{itemize}

\section{Multiple sequence alignment}

Understand the principle and main approach of multiple sequence alignment (MSA).

\begin{itemize}
    \item Understand different MSA scoring methods (sum of pairs, entropy-based).
    \item Understand why a dynamic programming solution exists in theory but is not practical.
    \item Understand and be able to apply the progressive multiple sequence alignment algorithm.
\end{itemize}

\section{Phylogenetics}

Understand what is phylogenetics and how phylogenetic trees are constructed.

\begin{itemize}
    \item Understand evolutionary relationships.
    \item Understand the difference between physiological trait-based and molecular sequence-based phylogenetics.
    \item Know the major assumptions of molecular phylogenetics.
    \item Know how to define the root of a tree using an outgroup or midpoint rooting.
    \item Know the different types of trees.
    \item Know the different steps in tree construction.
    \item Understand and be able to apply the different steps of the UPGMA algorithm for constructing a tree from a matrix of pairwise distances.
\end{itemize}

\end{document}